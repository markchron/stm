\chapter{io}
\section{subroutine structure}
The dataset is read in by two cycles. First, pre-read in the \emph{size-related} information. There include
\begin{itemize}
\item Geometric related (Nx,Ny,Nz) - formated related (Nine-point stencil or Five-point stencil)
\item Components and Phase related size - (component number, phase number, table size (viscosity table lines number, relative permeability lines number,
any table related input information needs the parameters)
\item well related info. (well numbers, well perforated layers number)
\item data (time) related info.  (time steps number) - dynamic related info needs
\end{itemize}
The pre-read information may also include permeabilities and well perforation information, which are important for transmissibilities and well index calculations. 
\subsection{Protocol}
After get the dataset name, the pre-read the size info, domain decomposition, then read in the total dataset. 

The data input process looks through the data on a natural ordering. 
\paragraph{pro:} Distribute the dataset (blocks(distributed) \& global control(replicated) \& global matrix structure(master) ). Global matrix structure depends on new block ordering techniques(D2, D4, red-black ...). The local matrix structure depends on the local block ordering. 
\subsection{Porotol1}
After get the dataset name, then pre-read the size info, and later read-in the whole information?
\paragraph{pro:} Look through the data file and check its validity
\paragraph{con:} The master processor may read in the whole information
\section{data structure}
The data type, 	integer (unsigned, signed - \emph{Fortran95 does not support unsigned integer}, integer);
				double precision (real(STDD) = real(8))
There are normally 3 types of data :
									scalars, vectors, mtx (mtx2d, mtxcsr)

